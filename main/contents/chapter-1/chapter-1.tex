\chapter{Perintah-perintah dasar}


\section{Penggunaan Sitasi}
Contoh penggunaan sitasi \cite{lukito2016,santosa2011user} \cite{lukito2016,santosa2011user,setiawan2014fuzzy,wibowo2014line}
\cite{setiawan2014fuzzy} \cite{wibowo2014line} \cite{marenda2016digitory} \cite{wibirama2013dual,wibowo2016clustering}

\section{Penulisan Gambar}

Contoh gambar terlihat pada Gambar \ref{Fig: Contoh gambar}. Gambar diambil dari \cite{wibowo2016clustering}.

\begin{figure}[h]
    \centering
    \includegraphics[width=10cm]{images/sample-fig.png}
    \caption{Contoh gambar.}
    \label{Fig: Contoh gambar}
\end{figure}

Contoh gambar terlihat pada Gambar \ref{fig:fig2}, \ref{fig:fig2a}, \ref{fig:fig2b}. Gambar diambil dari \cite{wibowo2016clustering}.

\begin{figure}[h]
    \centering
    \subfloat[]{
            \includegraphics[width=0.25\textwidth]{../main/images/sample-fig.png}  % .. means go up folder
            \label{fig:fig2a}}
        \hspace*{-1.2em}  % this is for giving horizontal space for fig 1 and fig 2.
    \subfloat[]{
            \includegraphics[width=0.25\textwidth]{../main/images/sample-fig.png}  % .. means go up folder
            \label{fig:fig2b}}
    \caption{(a) Figure 1 example and (b) Figure 2 example.}
    \label{fig:fig2}
\end{figure}

\section{Penulisan Tabel}
\begin{table}[h]
    \caption{tabel ini}
    \vspace{0.5em}
    \centering
    \begin{tabular}{|l|r|r|}
        \hline
        ID & Tinggi Badan (cm) & Berat Badan (kg) \\
        \hline \hline
        A23 & 173 & 62 \\
        A25 & 185 & 78 \\
        A10 & 162 & 70 \\ \hline
    \end{tabular}
    \label{Tab: Tabel Tinggi Berat}
\end{table}

Contoh penulisan tabel bisa dilihat pada Tabel \ref{Tab: Tabel Tinggi Berat}.

\begin{table}[h]
    \caption{tabel ini}
    \vspace{0.5em}
    \centering
    \begin{tabulary}{\linewidth}{|l|C|C|}
        \hline
        ID & Tinggi Badan Mahasiswa di Universitas Gadjah Mada (cm) & Berat Badan Mahasiswa di Universitas Gadjah Mada(kg) \\
        \hline \hline
        A23 &  \multicolumn{1}{r|}{173} &  \multicolumn{1}{r|}{62} \\
        A25 &  \multicolumn{1}{r|}{185} &  \multicolumn{1}{r|}{78} \\
        A10 &  \multicolumn{1}{r|}{162} &  \multicolumn{1}{r|}{70} \\ \hline
    \end{tabulary}
    \label{Tab: Tabel Tinggi Berat Kolom Panjang}
\end{table}

Contoh penulisan tabel dengan nama kolom sangat panjang dengan bantuan \textit{tabulary} bisa dilihat pada Tabel \ref{Tab: Tabel Tinggi Berat Kolom Panjang}.

\begin{table}[h]
  \centering
  \caption{Tabel Tinggi Berat 2}
  \vspace{0em}  % adjust caption and table space
  \begin{tabulary}{\textwidth}{lRRRR}  % no vertical line if not needed
    \toprule
    & \multicolumn{2}{c}{Laki-laki} & \multicolumn{2}{c}{Perempuan} \\
    \cline{2-5}
    \multirow{-2}{*}{ID} & Tinggi Badan (cm) & Berat Badan (kg) & Tinggi Badan (cm) & Berat Badan (kg)\\
    \hline
    A23 \cite{lukito2016} & 173           & 62          & 173           & 62          \\
    A25                   & \textbf{185}  & 70          & \textbf{185}  & 70          \\
    A10                   & 162           & \textbf{78} & 162           & \textbf{78} \\
    \cline{2-5}
    Total & 520 & 210 & 520 & 210 \\
    \bottomrule
  \end{tabulary}
  \label{tab:tinggiberat2}
\end{table}

Contoh penulisan tabel dengan gaya yang bersih bisa dilihat pada Tabel \ref{tab:tinggiberat2}.

\section{Penulisan formula}
Contoh penulisan formula
\begin{equation}
L_{\psi_z} = \{ t_i \mid v_z(t_i) \le \psi_z \}
\end{equation}

Contoh penulisan secara \textit{inline}: $\mathit{PV = nRT}$. Untuk kasus-kasus tertentu, kita membutuhkan perintah "mathit" dalam penulisan formula untuk menghindari adanya jeda saat penulisan formula.

Contoh formula \textbf{tanpa} menggunakan "mathit": $PVA = RTD$

Contoh formula \textbf{dengan} menggunakan "mathit": $\mathit{PVA = RTD}$

Untuk mengutip persamaan dapat juga menggunakan (\ref{eq:miqp-obj}). Persamaan dapat ditulis secara sentral pada file "equations/equations.tex".

    \inputeq{equations/equations}{miqp-obj}

\section{Contoh list}
Berikut contoh penggunaan list
\begin{enumerate}
    \item First item
    \item Second item
    \item Third item
\end{enumerate}

\section{Contoh \textit{highlight}}
Berikut contoh penggunaan \hly{\textit{highlight} kuning menggunakan} \verb|\hly{}|. Pengunaan \textit{highlight} tidak disarankan untuk dua paragraf sekaligus karena dapat mengakibatkan kerusakan dalam penomoran.

\hly{Gunakan perintah} \verb|\hly{}| yang terpisah untuk tiap paragrafnya seperti ini.

\hly{Penggunaan tanda \textit{curly bracket} \{ \} di dalam \textit{highlight} seperti penggunaan} \verb|\textit{}| \hlr{diperbolehkan, namun penghapusan otomatis oleh} \verb|"utilities/remove_highlight.ipynb"| \hly{tidak mendukung penghapusan \textit{nested bracket} seperti itu. Alternatifnya gunakan pemisahan} \textit{for the italic text} \hlg{seperti ini.}
