\documentclass{beamer}

\usepackage{hyperref}
\usetheme{metropolis}

\title{Guide for \texttt{thesisdtetiugm} Class File}
\subtitle{Equations in LaTeX}

\author{Muhammad Yasirroni}
\institute{Universitas Gadjah Mada}
\date{\today}

% NOTE: [fragile] required for \verb|...| and \begin{verbatim}

\begin{document}

\begin{frame}
  \titlepage
\end{frame}

\section{Introduction}
\begin{frame}
  \frametitle{What is LaTeX?}
  \begin{itemize}
    \item One of the strengths of using LaTeX compared to other editors is its excellent support for writing equations.
  \end{itemize}
\end{frame}

\section{Environments}
\begin{frame}[fragile]
  \frametitle{Various Environments for Equations}
  There are several environments in LaTeX that you can use to typeset equations. Here are some commonly used ones:

  \begin{itemize}
    \item inline equation: You can typeset equations inline within the text by enclosing them in single dollar signs (\$).

    \item \texttt{equation} environment: The \texttt{equation} environment is used for displaying standalone, single equation. It automatically numbers the equations and places them in the center of the line.

    \item \texttt{align} environment: The \texttt{align} environment is used to typeset multiline equations with alignment. It provides multiple alignment points using ampersands (\verb|&|).

    \item \texttt{align*} environment: The \texttt{align*} environment is similar to the \texttt{align} environment but does not automatically number the equations. It is useful when you don't want equation numbers.

  \end{itemize}
\end{frame}

\begin{frame}[fragile]
  \frametitle{Inline Equation}
  Inline equation is used when you want to mention equation or variable inside of sentence. For example, this equation of $E=mc^2$ is written using \verb|$E=mc^2$|. 
\end{frame}

\begin{frame}[fragile]
  \frametitle{\texttt{equation} Environment}

  \begin{verbatim}
\begin{equation}
    x = 2x + 3
\end{equation}
  \end{verbatim}

  \begin{equation}
      x = 2x + 3
  \end{equation}
\end{frame}

\begin{frame}[fragile]
  \frametitle{\texttt{align} Environment}

  Just like \texttt{equation} but support multiple equation and allign at \verb|&|.
  \begin{verbatim}
\begin{align}
  x &= 2x + 3 \\
  x - 2x &= 3 \\
  x &= 1
\end{align}
    \end{verbatim}

  \begin{align}
    x &= 2x + 3 \\
    x - 2x &= 3 \\
    x &= 1
  \end{align}
\end{frame}

\begin{frame}[fragile]
  \frametitle{\texttt{align*} Environment}

  Just like \texttt{align} but default to no numbering.
  \begin{verbatim}
\begin{align*}
  x &= 2x + 3 \\
  x - 2x &= 3 \\
  x &= 1
\end{align}
    \end{verbatim}

  \begin{align*}
    x &= 2x + 3 \\
    x - 2x &= 3 \\
    x &= 1
  \end{align}
\end{frame}

\begin{frame}[fragile]
  \frametitle{Equation Cotrol and Commands}
  \begin{itemize}
    \item \verb|\notag| to not add euqation numbering on current line of \texttt{align} environment. This is the recommended way.
    \item \verb|\tag{\stepcounter{equation}\theequation}| to add euqation numbering in \texttt{align*} environment.
  \end{itemize}
\end{frame}

\begin{frame}[fragile]
  \frametitle{Example Equation using \texttt{align} environment and \texttt{\textbackslash notag} \#1}

  \begin{verbatim}
\begin{align}
  f(x+h)          &= f(x)
                      + f^\prime(x)h
                      + \mathcal{O}(h^{2}) \notag \\
  f^\prime(x)h    &= f(x+h)
                      - f(x)
                      + \mathcal{O}(h^{2}) \notag \\
  f^\prime(x)     &= \frac{f(x+h) - f(x)}{h}
                      + \mathcal{O}(h)
\end{align}
  \end{verbatim}
\end{frame}

\begin{frame}[fragile]
  \frametitle{Example Equation using \texttt{align} environment and \texttt{\textbackslash notag} \#2}

\begin{align}
  f(x+h)          &= f(x)
                      + f^\prime(x)h
                      + \mathcal{O}(h^{2}) \notag \\
  -f^\prime(x)h    &= -f(x+h)
                      + f(x)
                      + \mathcal{O}(h^{2}) \notag \\
  f^\prime(x)     &= \frac{f(x+h) - f(x)}{h}
                      + \mathcal{O}(h)
  \end{align}
\end{frame}


\end{document}
