\documentclass{beamer}

\usepackage{hyperref}
\usetheme{metropolis}

\title{Guide for thesisdteti Class File}
\subtitle{Introduction to LaTeX}

\author{Muhammad Yasirroni}
\institute{Universitas Gadjah Mada}
\date{\today}

\begin{document}

\begin{frame}
  \titlepage
\end{frame}

\section{Introduction}
\begin{frame}
  \frametitle{What is LaTeX?}
  \begin{itemize}
    \item LaTeX is a document preparation system that allows users to create professional-looking documents with high-quality typography.
  \end{itemize}
\end{frame}

\begin{frame}
  \frametitle{Why using LaTeX?}
  \begin{itemize}
    \item Plain text file
    \item Separates content from formatting
    \item Great support for mathematical equation and symbols
  \end{itemize}
\end{frame}

\begin{frame}
  \frametitle{Where to Learn?}
  \begin{itemize}
    \item Learn LaTeX in 30 minutes by Overleaf \href{https://www.overleaf.com/learn/latex/Learn\_LaTeX\_in\_30\_minutes}{here}.
    \item Learn LaTeX by learnlatex.org \href{https://www.learnlatex.org/en/}{here}.
  \end{itemize}
\end{frame}

\section{Getting Started}
\begin{frame}[fragile]
  \frametitle{Minimal Working Example}
  Create a new project in Overleaf \href{https://www.overleaf.com/project}{here}, and type:

  \begin{block}{}
    \vspace{-2em}
    \scriptsize
    \begin{verbatim}
          \documentclass{report}
          \renewcommand{\thesection}{\arabic{section}}

          \title{Minimal Working Example}
          \author{Muhammad Yasirroni}
          \date{\today}

          \begin{document}
          \maketitle

          \section{Introduction}
          \subsection{Background}
          Hello world! This is a simple example!

          \subsection{Conclusion}
          See you!

          \end{document}
      \end{verbatim}
  \end{block}

\end{frame}

\begin{frame}
  \frametitle{Document Structure}
  \begin{itemize}
    \item Creating sections and subsections.
    \item Adding figures and tables.
    \item Creating a bibliography.
  \end{itemize}
\end{frame}

\begin{frame}
  \frametitle{Formatting}
  \begin{itemize}
    \item Different font styles and sizes.
    \item Creating lists (ordered and unordered).
    \item Adding footnotes and endnotes.
    \item Creating mathematical equations and symbols.
  \end{itemize}
\end{frame}

\begin{frame}
  \frametitle{Customization}
  \begin{itemize}
    \item Changing the document class and layout.
    \item Using packages to add additional functionality.
    \item Creating custom commands and environments.
  \end{itemize}
\end{frame}

\begin{frame}
  \frametitle{Conclusion}
  \begin{itemize}
    \item Summary of the main points of the lecture.
    \item Additional resources for learning LaTeX.
    \item Encouragement to practice and experiment with LaTeX.
  \end{itemize}
\end{frame}

\end{document}
