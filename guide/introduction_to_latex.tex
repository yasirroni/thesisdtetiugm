\documentclass{beamer}

\usepackage{hyperref}
\usetheme{metropolis}

\title{Guide for \texttt{thesisdteti} Class File}
\subtitle{Introduction to LaTeX}

\author{Muhammad Yasirroni}
\institute{Universitas Gadjah Mada}
\date{\today}

% NOTE: [fragile] required for \verb|...| and \begin{verbatim}

\begin{document}

\begin{frame}
  \titlepage
\end{frame}

\section{Introduction}
\begin{frame}
  \frametitle{What is LaTeX?}
  \begin{itemize}
    \item LaTeX is a document preparation system that allows users to create professional-looking documents with high-quality typography.
    \item LaTeX document in \texttt{.tex} file can be rendered into portable \texttt{.pdf} file. This rendering process supports a formatter in the form of \texttt{.cls} document class and \texttt{.bst} bibliography style document.
  \end{itemize}
\end{frame}

\begin{frame}
  \frametitle{Why using LaTeX?}
  \begin{itemize}
    \item Plain text file
    \item Separates content from formatting
    \item Great support for mathematical equation and symbols
  \end{itemize}
\end{frame}

\begin{frame}
  \frametitle{Where to Learn?}
  \begin{itemize}
    \item Learn LaTeX in 30 minutes by Overleaf \href{https://www.overleaf.com/learn/latex/Learn\_LaTeX\_in\_30\_minutes}{here}.
    \item Learn LaTeX by learnlatex.org \href{https://www.learnlatex.org/en/}{here}.
  \end{itemize}
\end{frame}

\section{Getting Started}
\begin{frame}[fragile]
  \frametitle{Minimal Working Example}
  Create a new project in Overleaf (\href{https://www.overleaf.com/project}{link}), and type:

  \begin{block}{}
    \vspace{-2em}
    \scriptsize
    \begin{verbatim}
      \documentclass{report}

      \title{Minimal Working Example}
      \author{Muhammad Yasirroni}
      
      \begin{document}
      \maketitle
      
      \chapter{Introduction}
      \section{Background}
      \subsection{Related Works}
      Hello world! This is a simple example!
      
      \subsection{Conclusion}
      See you!
      
      \end{document}
      \end{verbatim}
  \end{block}
\end{frame}

\begin{frame}[fragile]
\frametitle{Using thesisdtetiugm.cls}
\begin{block}{}
  \vspace{-2em}
  \scriptsize
  \begin{verbatim}
    % change documentclass to use thesisdtetiugm.cls
    \documentclass[master,bahasa,table,xcdraw]{thesisdtetiugm}

    \title{Minimal Working Example}
    \author{Muhammad Yasirroni}{<<NIM>>}  % add NIM
    
    % minimal data for cover
    \program{<<Program name>>}{<<Program coordinator>>}{<<NIP>>}
    \major{<<Major>>}
    \yearsubmit{<<year submit>>}
    
    \begin{document}
    \printcover{images/logougm.png}{Tesis}  % replacing \maketitle
    
    \chapter{Introduction}
    \section{Background}
    \subsection{Related Works}
    Hello world! This is a simple example!
    
    \subsection{Conclusion}
    See you!
    
    \end{document}
    \end{verbatim}
\end{block}
\end{frame}

\section{Style Using Document Class}
\begin{frame}[fragile]
  \frametitle{Document Class}

  Template or a blueprint for your document. It determines the overall layout, formatting, and structure of your document. To use it, use \verb|\documentclass{}| syntax. Some built-in document class example are:

  \begin{itemize}
    \item article: short documents, such as papers or reports
    \item report: longer documents, such as theses or technical reports
    \item book: for writing books with two kinds of pages and various book supports 
    \item beamer: presentation files with slides
  \end{itemize}
\end{frame}

\begin{frame}
  \frametitle{Custom Document Class}

  Publisher often provide their own document class in \texttt{.cls} file format. Some custom document class are:

  \begin{itemize}
    \item thesisdtetiugm: UGM thesis style for various degrees
    \item IEEEtran: IEEE Transaction style
    \item ieeeaccess: IEEE Access style
    \item elsarticle: Elsevier style
  \end{itemize}
\end{frame}

\section{Benefit of Using LaTeX}
\begin{frame}[fragile]
  \frametitle{Benefit of Using LaTeX \#1}

  \textbf{Mixing Style}
  
  Even when you use \texttt{.cls} styler document, you still can add your own style inside your \texttt{.tex} file. Both \texttt{.cls} and \texttt{.tex} use same syntax and (in most cases) you can think of \verb|\documentclass{}| as "load all the text in \texttt{.cls} to the \texttt{.tex} file."

  To create a new command, use \verb|\newcommand| and use \verb|\renewcommand| to redefine existing command that is already define previously, including in \texttt{.cls} styler document.

\end{frame}

\begin{frame}
  \frametitle{Benefit of Using LaTeX \#2}

  \textbf{Code}
  
  Whenever you update your code, your output \texttt{.pdf} file will also get updated upon rendering. Thus you can take advantage of it by:

  \begin{itemize}
    \item Place your simulation code under \texttt{codes/} folder.
    \item Output the graph of your simulation directly to \texttt{images/} folder in \texttt{.png} or \texttt{.pdf} file.
  \end{itemize}

\end{frame}

\begin{frame}
  \frametitle{Benefit of Using LaTeX \#3}

  \textbf{Tables and Figures}
  
  Tables and figures are automatically numbered and placed on the right place. Very useful for journal publication where it require every tables and figures to be placed after mentioned.

\end{frame}

\begin{frame}
  \frametitle{Benefit of Using LaTeX \#4}

  \textbf{References}

  One file to store references in \texttt{.bib} file that can be written manually or imported from various references manages such as Mendeley, Zotero, and Semantic Scholar.

  The citation style can be easily formatted using \texttt{.bst} file, make it easy to change style if your paper got rejected.

\end{frame}

\begin{frame}
  \frametitle{Benefit of Using LaTeX \#5}

  \textbf{Plain Text}

  \texttt{.tex} file is a plain file that can take advantages of:

  \begin{itemize}
    \item Place your simulation code under \texttt{codes/} folder.
    \item Output the graph of your simulation directly to \texttt{images/} folder in \texttt{.png} or \texttt{.pdf} file.
  \end{itemize}

\end{frame}

% \begin{frame}
%   \frametitle{Document Structure}
%   \begin{itemize}
%     \item Creating sections and subsections.
%     \item Adding figures and tables.
%     \item Creating a bibliography.
%   \end{itemize}
% \end{frame}

% \begin{frame}
%   \frametitle{Formatting}
%   \begin{itemize}
%     \item Different font styles and sizes.
%     \item Creating lists (ordered and unordered).
%     \item Adding footnotes and endnotes.
%     \item Creating mathematical equations and symbols.
%   \end{itemize}
% \end{frame}

% \begin{frame}
%   \frametitle{Customization}
%   \begin{itemize}
%     \item Changing the document class and layout.
%     \item Using packages to add additional functionality.
%     \item Creating custom commands and environments.
%   \end{itemize}
% \end{frame}

% \begin{frame}
%   \frametitle{Conclusion}
%   \begin{itemize}
%     \item Summary of the main points of the lecture.
%     \item Additional resources for learning LaTeX.
%     \item Encouragement to practice and experiment with LaTeX.
%   \end{itemize}
% \end{frame}

\end{document}
